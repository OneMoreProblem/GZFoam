%  LaTeX support: latex@mdpi.com 
%  For support, please attach all files needed for compiling as well as the log file, and specify your operating system, LaTeX version, and LaTeX editor.

%=================================================================
\documentclass[sensors,article,submit,moreauthors,pdftex]{Definitions/mdpi} 

% For posting an early version of this manuscript as a preprint, you may use "preprints" as the journal and change "submit" to "accept". The document class line would be, e.g., \documentclass[preprints,article,accept,moreauthors,pdftex]{mdpi}. This is especially recommended for submission to arXiv, where line numbers should be removed before posting. For preprints.org, the editorial staff will make this change immediately prior to posting.

%--------------------
% Class Options:
%--------------------
%----------
% journal
%----------
% Choose between the following MDPI journals:
% acoustics, actuators, addictions, admsci, adolescents, aerospace, agriculture, agriengineering, agronomy, ai, algorithms, allergies, analytica, animals, antibiotics, antibodies, antioxidants, appliedchem, applmech, applmicrobiol, applnano, applsci, arts, asi, atmosphere, atoms, audiolres, automation, axioms, batteries, bdcc, behavsci, beverages, biochem, bioengineering, biologics, biology, biomechanics, biomedicines, biomedinformatics, biomimetics, biomolecules, biophysica, biosensors, biotech, birds, bloods, brainsci, buildings, businesses, cancers, carbon, cardiogenetics, catalysts, cells, ceramics, challenges, chemengineering, chemistry, chemosensors, chemproc, children, civileng, cleantechnol, climate, clinpract, clockssleep, cmd, coatings, colloids, compounds, computation, computers, condensedmatter, conservation, constrmater, cosmetics, crops, cryptography, crystals, curroncol, cyber, dairy, data, dentistry, dermato, dermatopathology, designs, diabetology, diagnostics, digital, disabilities, diseases, diversity, dna, drones, dynamics, earth, ebj, ecologies, econometrics, economies, education, ejihpe, electricity, electrochem, electronicmat, electronics, encyclopedia, endocrines, energies, eng, engproc, entropy, environments, environsciproc, epidemiologia, epigenomes, fermentation, fibers, fire, fishes, fluids, foods, forecasting, forensicsci, forests, fractalfract, fuels, futureinternet, futuretransp, futurepharmacol, futurephys, galaxies, games, gases, gastroent, gastrointestdisord, gels, genealogy, genes, geographies, geohazards, geomatics, geosciences, geotechnics, geriatrics, hazardousmatters, healthcare, hearts, hemato, heritage, highthroughput, histories, horticulturae, humanities, hydrogen, hydrology, hygiene, idr, ijerph, ijfs, ijgi, ijms, ijns, ijtm, ijtpp, immuno, informatics, information, infrastructures, inorganics, insects, instruments, inventions, iot, j, jcdd, jcm, jcp, jcs, jdb, jfb, jfmk, jimaging, jintelligence, jlpea, jmmp, jmp, jmse, jne, jnt, jof, joitmc, jor, journalmedia, jox, jpm, jrfm, jsan, jtaer, jzbg, kidney, land, languages, laws, life, liquids, literature, livers, logistics, lubricants, machines, macromol, magnetism, magnetochemistry, make, marinedrugs, materials, materproc, mathematics, mca, measurements, medicina, medicines, medsci, membranes, metabolites, metals, metrology, micro, microarrays, microbiolres, micromachines, microorganisms, minerals, mining, modelling, molbank, molecules, mps, mti, nanoenergyadv, nanomanufacturing, nanomaterials, ncrna, network, neuroglia, neurolint, neurosci, nitrogen, notspecified, nri, nursrep, nutrients, obesities, oceans, ohbm, onco, oncopathology, optics, oral, organics, osteology, oxygen, parasites, parasitologia, particles, pathogens, pathophysiology, pediatrrep, pharmaceuticals, pharmaceutics, pharmacy, philosophies, photochem, photonics, physchem, physics, physiolsci, plants, plasma, pollutants, polymers, polysaccharides, proceedings, processes, prosthesis, proteomes, psych, psychiatryint, publications, quantumrep, quaternary, qubs, radiation, reactions, recycling, regeneration, religions, remotesensing, reports, reprodmed, resources, risks, robotics, safety, sci, scipharm, sensors, separations, sexes, signals, sinusitis, smartcities, sna, societies, socsci, soilsystems, solids, sports, standards, stats, stresses, surfaces, surgeries, suschem, sustainability, symmetry, systems, taxonomy, technologies, telecom, textiles, thermo, tourismhosp, toxics, toxins, transplantology, traumas, tropicalmed, universe, urbansci, uro, vaccines, vehicles, vetsci, vibration, viruses, vision, water, wevj, women, world 

%---------
% article
%---------
% The default type of manuscript is "article", but can be replaced by: 
% abstract, addendum, article, book, bookreview, briefreport, casereport, comment, commentary, communication, conferenceproceedings, correction, conferencereport, entry, expressionofconcern, extendedabstract, datadescriptor, editorial, essay, erratum, hypothesis, interestingimage, obituary, opinion, projectreport, reply, retraction, review, perspective, protocol, shortnote, studyprotocol, systematicreview, supfile, technicalnote, viewpoint, guidelines, registeredreport, tutorial
% supfile = supplementary materials

%----------
% submit
%----------
% The class option "submit" will be changed to "accept" by the Editorial Office when the paper is accepted. This will only make changes to the frontpage (e.g., the logo of the journal will get visible), the headings, and the copyright information. Also, line numbering will be removed. Journal info and pagination for accepted papers will also be assigned by the Editorial Office.

%------------------
% moreauthors
%------------------
% If there is only one author the class option oneauthor should be used. Otherwise use the class option moreauthors.

%---------
% pdftex
%---------
% The option pdftex is for use with pdfLaTeX. If eps figures are used, remove the option pdftex and use LaTeX and dvi2pdf.

%%%% ДЛЯ РУССКОГО ТЕКСТА закомментировать потом!
\usepackage{inputenc}
\usepackage[T2A,T1]{fontenc}
\usepackage[english,russian]{babel}
\usepackage{cmap}
%%%%

%=================================================================
% MDPI internal commands
\firstpage{1} 
\makeatletter 
\setcounter{page}{\@firstpage} 
\makeatother
\pubvolume{1}
\issuenum{1}
\articlenumber{0}
\pubyear{2021}
\copyrightyear{2020}
%\externaleditor{Academic Editor: Firstname Lastname} % For journal Automation, please change Academic Editor to "Communicated by"
\datereceived{} 
\dateaccepted{} 
\datepublished{} 
\hreflink{https://doi.org/} % If needed use \linebreak
%------------------------------------------------------------------
% The following line should be uncommented if the LaTeX file is uploaded to arXiv.org
%\pdfoutput=1

%=================================================================
% Add packages and commands here. The following packages are loaded in our class file: fontenc, inputenc, calc, indentfirst, fancyhdr, graphicx, epstopdf, lastpage, ifthen, lineno, float, amsmath, setspace, enumitem, mathpazo, booktabs, titlesec, etoolbox, tabto, xcolor, soul, multirow, microtype, tikz, totcount, changepage, paracol, attrib, upgreek, cleveref, amsthm, hyphenat, natbib, hyperref, footmisc, url, geometry, newfloat, caption

%=================================================================
%% Please use the following mathematics environments: Theorem, Lemma, Corollary, Proposition, Characterization, Property, Problem, Example, ExamplesandDefinitions, Hypothesis, Remark, Definition, Notation, Assumption
%% For proofs, please use the proof environment (the amsthm package is loaded by the MDPI class).

%=================================================================
% Full title of the paper (Capitalized)
\Title{Title}

% MDPI internal command: Title for citation in the left column
\TitleCitation{Title}

% Author Orchid ID: enter ID or remove command
\newcommand{\orcidauthorA}{0000-0001-5273-2471} % Add \orcidA{} behind the author's name
\newcommand{\orcidauthorB}{0000-0002-8736-0652} % Add \orcidB{} behind the author's name
%\newcommand{\orcidauthorC}{0000-0002-4013-2329} % Add \orcidB{} behind the author's name

% Authors, for the paper (add full first names)
\Author{Konstantin Barkalov $^{1}$*\orcidA{} and Ilya Lebedev $^{1}$\orcidB{}}

% MDPI internal command: Authors, for metadata in PDF
\AuthorNames{Konstantin Barkalov and Ilya Lebedev}

% MDPI internal command: Authors, for citation in the left column
\AuthorCitation{Barkalov, K.; Lebedev, I.}
% If this is a Chicago style journal: Lastname, Firstname, Firstname Lastname, and Firstname Lastname.

% Affiliations / Addresses (Add [1] after \address if there is only one affiliation.)
\address{%
$^{1}$ \quad Department of Mathematical Software and Supercomputing Technologies, Lobachevsky University, 603950 Nizhni Novgorod, Russia; ilya.lebedev@itmm.unn.ru (I.L.)\\
$^{2}$ \quad Affiliation 2; e-mail@e-mail.com}

% Contact information of the corresponding author
\corres{Correspondence: e-mail@e-mail.com; Tel.: (optional; include country code; if there are multiple corresponding authors, add author initials) +xx-xxxx-xxx-xxxx (F.L.)}

% Current address and/or shared authorship
%\firstnote{Current address: Affiliation 3} 
%\secondnote{These authors contributed equally to this work.}
% The commands \thirdnote{} till \eighthnote{} are available for further notes

%\simplesumm{} % Simple summary

%\conference{} % An extended version of a conference paper

% Abstract (Do not insert blank lines, i.e. \\) 
\abstract{A single paragraph of about 200 words maximum. For research articles, abstracts should give a pertinent overview of the work. We strongly encourage authors to use the following style of structured abstracts, but without headings: (1) Background: place the question addressed in a broad context and highlight the purpose of the study; (2) Methods: describe briefly the main methods or treatments applied; (3) Results: summarize the article's main findings; (4) Conclusion: indicate the main conclusions or interpretations. The abstract should be an objective representation of the article, it must not contain results which are not presented and substantiated in the main text and should not exaggerate the main conclusions.}

% Keywords
\keyword{keyword 1; keyword 2; keyword 3 (List three to ten pertinent keywords specific to the article; yet reasonably common within the subject discipline.)} 

% The fields PACS, MSC, and JEL may be left empty or commented out if not applicable
%\PACS{J0101}
%\MSC{}
%\JEL{}

%%%%%%%%%%%%%%%%%%%%%%%%%%%%%%%%%%%%%%%%%%
% Only for the journal Diversity
%\LSID{\url{http://}}

%%%%%%%%%%%%%%%%%%%%%%%%%%%%%%%%%%%%%%%%%%
% Only for the journal Applied Sciences:
%\featuredapplication{Authors are encouraged to provide a concise description of the specific application or a potential application of the work. This section is not mandatory.}
%%%%%%%%%%%%%%%%%%%%%%%%%%%%%%%%%%%%%%%%%%

%%%%%%%%%%%%%%%%%%%%%%%%%%%%%%%%%%%%%%%%%%
% Only for the journal Data:
%\dataset{DOI number or link to the deposited data set in cases where the data set is published or set to be published separately. If the data set is submitted and will be published as a supplement to this paper in the journal Data, this field will be filled by the editors of the journal. In this case, please make sure to submit the data set as a supplement when entering your manuscript into our manuscript editorial system.}

%\datasetlicense{license under which the data set is made available (CC0, CC-BY, CC-BY-SA, CC-BY-NC, etc.)}

%%%%%%%%%%%%%%%%%%%%%%%%%%%%%%%%%%%%%%%%%%
% Only for the journal Toxins
%\keycontribution{The breakthroughs or highlights of the manuscript. Authors can write one or two sentences to describe the most important part of the paper.}

%%%%%%%%%%%%%%%%%%%%%%%%%%%%%%%%%%%%%%%%%%
% Only for the journal Encyclopedia
%\encyclopediadef{Instead of the abstract}
%\entrylink{The Link to this entry published on the encyclopedia platform.}
%%%%%%%%%%%%%%%%%%%%%%%%%%%%%%%%%%%%%%%%%%

\begin{document}
%%%%%%%%%%%%%%%%%%%%%%%%%%%%%%%%%%%%%%%%%%
\section{Introduction}

Во введении -- охарактеризовать проблему, дать обзор методов ее решения.


Проблема математического моделирования опасных природных явлений и процессов помимо использования сложных моделей (что определяет общую структуру проводимого исследования) включает в себя также выбор наилучших значений параметров используемых моделей. Выбор конкретных значений параметров (при зафиксированной модели) может существенным образом влиять на качество моделирования. 

Сложность исследуемых явлений и процессов отражается в сложности соответствующих математических моделей и численных методов их анализа. В настоящее время основным (и зачастую единственно возможным) инструментом подобного анализа является суперкомпьютерное моделирование поведения объекта, для чего широко используются автоматизированные CAD/CAM/CAE-системы. Известным примером открытого программного обеспечения подобного рода является OpenFOAM -- открытая платформа для численного моделирования задач механики сплошных сред [ссылка]). 

Рост вычислительных мощностей современных суперкомпьютерных систем идет параллельно с усложнением математических моделей рассматриваемых процессов, что делает проведение одного модельного расчета (a \textit{trial}) трудоемкой операцией. Например, для адекватного моделирования [здесь нужен пример сложной модели] требуется [пример использования ресурсов] . Следовательно, выбор наилучших значений параметров модели за приемлемое время не может быть выполнен перебором всех возможных вариантов методом «проб и ошибок», т.е. перебором на некоторой регулярной сетке в области изменения параметров.
Невозможность проведения большого числа поисковых испытаний требует применения эффективных поисковых алгоритмов, которые при проведении относительно небольшого числа испытаний давали бы приемлемую оценку решения задачи в рамках доступного вычислительного ресурса.

Следует отметить, что у большинства существующих методов поиска минимума time-consuming black-box functions есть свои недостатки. Gradient-based algorithms не могут использоваться во многих случаях просто потому, что производные целевой функции не известны, а их конечно-разностные аппроксимации слишком дороги для вычисления. Одновременно с этим методы на основе градиентов обеспечивают поиск, вообще говоря, лишь локально оптимального решения задачи.
Классические методы прямого поиска, которые не требуют использования производных, такие как Nelder-Mead method [] or Hooke-Jeeves method [], также являются локальными. Как правило, применение указанных методов для решения задач глобальной оптимизации включает несколько рестартов из узлов случайной сетки, что требует большого числа испытаний. 

Детерминированные методы липшицевой глобальной оптимизации, такие как DIRECT [], метод неравномерных покрытий [], диагональные [] и симплексные [] методы гарантируют (в пределе) сходимость к глобальному решению задачи, но могут потребовать большого числа поисковых испытаний.
Наконец, эвристические методы, такие как эволюционные алгоритмы и имитация отжига, также требуют очень большого количества вычислений функций для получения хороших оценок решений в задачах глобальной оптимизации, и одновременно проигрывают в качестве  детерминированным алгоритмам [].

Таким образом, применение оптимизационных алгоритмов к поиску минимума непосредственно time-consuming function может оказаться чрезмерно дорогим, если проведение одного испытания занимает много машинного времени.
 Известным способом преодоления этой проблемы является построение аппроксимации целевой функции (also known as a response surface model, a metamodel or a surrogate model), вычисление значений которой является дешевой операцией, и дальнейшее использование этой аппроксимации для поиска минимума. 
Существует множество вариантов построения аппроксимаций для multivariate functions. К ним относятся, например, различные методы методы интерполяции, использующие полиномы, splines, radial basis functions, Kriging, а также различные регрессионные модели. Многие из данных алгоритмов используются для построения методов глобальной оптимизации. 
Например, в [] детально проработано использование radial basis functions. В [] предлагаются Kriging-based methods. Оригинальный подход к построению метамоделей и trust-region методов на их основе представлен в [Торопов]. 

В проведенном исследовании мы использовали эффективный global search algorithm [] для решения задач липшицевой глобальной оптимизации в сочетании с аппроксимацией функции на основе регрессионных моделей. На первой стадии поиска GSA работал как прямой метод непосредственно с целевой функцией. Затем по накопленной информации строилась аппроксимация целевой функции, которая использовалась на второй стадии поиска.
Для построения аппроксимаций мы применяли Support Vector Regression (SVR) and Neural Network Regression (NNR). 
Численные эксперименты показали ...

The main part of the paper has the following structure. 
Section 2 contains a description of the ... 
Section 3 describes ... 
In Section 4, ways of ... are discussed. 
Section 5 contains the results of numerical experiments. 
Section 6 concludes the paper.




%%%%%%%%%%%%%%%%%%%%%%%%%%%%%%%%%%%%%%%%%%
\section{Problem Statement}

%ИСП
Содержательная постановка задачи -- описание исследуемого объекта, его параметров. 
Описание математической модели объекта как ДУЧП
Параметры дифференциального уравнения
Подчеркнуть важность исследования и сложность анализа математической модели объекта

%ННГУ

Будем предполагать, что выбор того или иного набора значений параметров модели определяется значениями вектора $y=(y_1,y_2,...,y_N)$, а качество модели, соответствующие заданному значению вектора параметров, описывается функцией $\varphi(y)$. Будем называть данную функцию \textit{критерием оптимизации}, причем уменьшение значения критерия соответствует более хорошей математической модели. Также будем предполагать, что должны выполняться некоторые требования, гарантирующие применимость модели. Выполнение указанных требований обычно формулируется как условие принадлежности вектора $y$ гиперинтервалу $D$,
\[
D={a_i \leq y_i \leq b_i, \; 1 \leq i \leq N}.
\]

Таким образом, процессу выбора оптимального набора параметров модели соответствует \textit{задача глобальной оптимизации} вида
\begin{eqnarray}\label{main_problem}
& \varphi(y^\ast)=\min{\left\{\varphi(y):y\in D\right\}},\\
& D=\left\{y\in \text{R}^N: a_i\leq y_i \leq b_i, 1\leq i \leq N\right\}. \nonumber
\end{eqnarray}

Рассматриваемые задачи характеризуются тем фактом, что целевая функция $\varphi(y)$ не задана аналитически; есть лишь некоторый алгоритм вычисления ее значений в точках области $D$. При этом одно поисковое испытание соответствует одному расчету по модели и является вычислительно-трудоемкой операцией [ссылки].

Задачи многоэкстремальной оптимизации имеют существенно более высокую трудоемкость решения по сравнению с другими типами оптимизационных задач, т.к. глобальный оптимум является интегральной характеристикой решаемой задачи и требует исследования всей области поиска. Как результат, поиск глобального оптимума сводится к построению некоторого покрытия (сетки) в области параметров, и выборе наилучшего значения функции на данной сетке. Снижение объема вычислений может быть достигнуто при построении неравномерного покрытия области поиска: сетка должна быть достаточно плотной в окрестности глобального оптимума и более редкой вдали от искомого решения.

Типичным предположением, которое используют многие методы глобальной оптимизации [ссылки], является предположение о том, что целевая функция $\varphi(y)$ satisfies the Lipschitz condition
\[
\left|\varphi(y_1)-\varphi(y_2)\right|\leq L\left\|y_1-y_2\right\|,\; y_1,y_2 \in D, 0<L<\infty.
\]
Предположение такого рода является достаточно естественным для многих прикладных задач, поскольку относительные вариации функции, характеризующей моделируемый процесс, обычно не могут превышать некоторый порог, определяемый ограниченной энергией изменений. Возникающий при этом вопрос об оценке априори неизвестных значений Lipschitz constant может решаться путем введения адаптивных схем (см., например, работы [ссылки]).





 
%%%%%%%%%%%%%%%%%%%%%%%%%%%%%%%%%%%%%%%%%%
\section{Methods}
%ИСП
\subsection{Используемый численный метод решения ДУЧП}

\subsection{}

%ННГУ
\subsection{Метод глобального поиска}

\subsection{Построение аппроксимаций целевой функции}

\subsection{Использование аппроксимаций при решении задачи оптимизации}



%%%%%%%%%%%%%%%%%%%%%%%%%%%%%%%%%%%%%%%%%%
\section{Results}

Описание оборудования и программного обеспечения, которое было задействовано при проведении экспериментов.

Результаты расчетов

Иллюстрации

Сравнение нескольких разных решений (это можно перенести и в следующую секцию)

%This section may be divided by subheadings. It should provide a concise and precise description of the experimental results, their interpretation as well as the experimental conclusions that can be drawn.
%\subsection{Subsection}
%\subsubsection{Subsubsection}
%
%Bulleted lists look like this:
%\begin{itemize}
%\item	First bullet;
%\item	Second bullet;
%\item	Third bullet.
%\end{itemize}
%
%Numbered lists can be added as follows:
%\begin{enumerate}
%\item	First item; 
%\item	Second item;
%\item	Third item.
%\end{enumerate}
%
%The text continues here. 
%
%\subsection{Figures, Tables and Schemes}
%
%All figures and tables should be cited in the main text as Figure~\ref{fig1}, Table~\ref{tab1}, etc.
%
%\begin{figure}[H]
%\includegraphics[width=10.5 cm]{Definitions/logo-mdpi}
%\caption{This is a figure. Schemes follow the same formatting. If there are multiple panels, they should be listed as: (\textbf{a}) Description of what is contained in the first panel. (\textbf{b}) Description of what is contained in the second panel. Figures should be placed in the main text near to the first time they are cited. A caption on a single line should be centered.\label{fig1}}
%\end{figure}   
%
%% The MDPI table float is called specialtable
%\begin{specialtable}[H] 
%\caption{This is a table caption. Tables should be placed in the main text near to the first time they are~cited.\label{tab1}}
%%%% \tablesize{} %% You can specify the fontsize here, e.g., \tablesize{\footnotesize}. If commented out \small will be used.
%\begin{tabular}{ccc}
%\toprule
%\textbf{Title 1}	& \textbf{Title 2}	& \textbf{Title 3}\\
%\midrule
%Entry 1		& Data			& Data\\
%Entry 2		& Data			& Data\\
%\bottomrule
%\end{tabular}
%\end{specialtable}
%
%%\begin{listing}[H]
%%\caption{Title of the listing}
%%\rule{\columnwidth}{1pt}
%%\raggedright Text of the listing. In font size footnotesize, small, or normalsize. Preferred format: left aligned and single spaced. Preferred border format: top border line and bottom border line.
%%\rule{\columnwidth}{1pt}
%%\end{listing}
%
%Text.
%
%Text.
%
%\subsection{Formatting of Mathematical Components}
%
%This is the example 1 of equation:
%\begin{equation}
%a = 1,
%\end{equation}
%the text following an equation need not be a new paragraph. Please punctuate equations as regular text.
%%% If the documentclass option "submit" is chosen, please insert a blank line before and after any math environment (equation and eqnarray environments). This ensures correct linenumbering. The blank line should be removed when the documentclass option is changed to "accept" because the text following an equation should not be a new paragraph.
%
%This is the example 2 of equation:
%\end{paracol}
%\nointerlineskip
%\begin{equation}
%a = b + c + d + e + f + g + h + i + j + k + l + m + n + o + p + q + r + s + t + u + v + w + x + y + z
%\end{equation}
%
%% Example of a figure that spans the whole page width (the commands \widefigure and \begin{paracol}{2}, \linenumbers, and\switchcolumn need to be present). The same concept works for tables, too.
%\begin{figure}[H]	
%\widefigure
%\includegraphics[width=15 cm]{Definitions/logo-mdpi}
%\caption{This is a wide figure.\label{fig2}}
%\end{figure}  
%\begin{paracol}{2}
%\linenumbers
%\switchcolumn
%
%Please punctuate equations as regular text. Theorem-type environments (including propositions, lemmas, corollaries etc.) can be formatted as follows:
%%% Example of a theorem:
%\begin{Theorem}
%Example text of a theorem.
%\end{Theorem}
%
%The text continues here. Proofs must be formatted as follows:
%
%%% Example of a proof:
%\begin{proof}[Proof of Theorem 1]
%Text of the proof. Note that the phrase ``of Theorem 1'' is optional if it is clear which theorem is being referred to.
%\end{proof}
%The text continues here.

%%%%%%%%%%%%%%%%%%%%%%%%%%%%%%%%%%%%%%%%%%
\section{Discussion}

Authors should discuss the results and how they can be interpreted from the perspective of previous studies and of the working hypotheses. The findings and their implications should be discussed in the broadest context possible. Future research directions may also be highlighted.

%%%%%%%%%%%%%%%%%%%%%%%%%%%%%%%%%%%%%%%%%%
\section{Conclusions}

This section is not mandatory, but can be added to the manuscript if the discussion is unusually long or complex.

%%%%%%%%%%%%%%%%%%%%%%%%%%%%%%%%%%%%%%%%%%
%\section{Patents}
%
%This section is not mandatory, but may be added if there are patents resulting from the work reported in this manuscript.

%%%%%%%%%%%%%%%%%%%%%%%%%%%%%%%%%%%%%%%%%%
\vspace{6pt} 

%%%%%%%%%%%%%%%%%%%%%%%%%%%%%%%%%%%%%%%%%%
%% optional
%\supplementary{The following are available online at \linksupplementary{s1}, Figure S1: title, Table S1: title, Video S1: title.}

% Only for the journal Methods and Protocols:
% If you wish to submit a video article, please do so with any other supplementary material.
% \supplementary{The following are available at \linksupplementary{s1}, Figure S1: title, Table S1: title, Video S1: title. A supporting video article is available at doi: link.} 

%%%%%%%%%%%%%%%%%%%%%%%%%%%%%%%%%%%%%%%%%%
\authorcontributions{For research articles with several authors, a short paragraph specifying their individual contributions must be provided. The following statements should be used ``Conceptualization, X.X. and Y.Y.; methodology, X.X.; software, X.X.; validation, X.X., Y.Y. and Z.Z.; formal analysis, X.X.; investigation, X.X.; resources, X.X.; data curation, X.X.; writing---original draft preparation, X.X.; writing---review and editing, X.X.; visualization, X.X.; supervision, X.X.; project administration, X.X.; funding acquisition, Y.Y. All authors have read and agreed to the published version of the manuscript.'', please turn to the  \href{http://img.mdpi.org/data/contributor-role-instruction.pdf}{CRediT taxonomy} for the term explanation. Authorship must be limited to those who have contributed substantially to the work~reported.}

\funding{This research was funded by the Ministry of Science and Higher Education of the Russian Federation, agreement number 075-15-2020-808.}

\institutionalreview{Not applicable.}

\informedconsent{Not applicable.}

\dataavailability{In this section, please provide details regarding where data supporting reported results can be found, including links to publicly archived datasets analyzed or generated during the study. Please refer to suggested Data Availability Statements in section ``MDPI Research Data Policies'' at \url{https://www.mdpi.com/ethics}. You might choose to exclude this statement if the study did not report any data.} 

%\acknowledgments{In this section you can acknowledge any support given which is not covered by the author contribution or funding sections. This may include administrative and technical support, or donations in kind (e.g., materials used for experiments).}

\conflictsofinterest{The authors declare no conflict of interest.} 

%% Optional
%\sampleavailability{Samples of the compounds ... are available from the authors.}

%%%%%%%%%%%%%%%%%%%%%%%%%%%%%%%%%%%%%%%%%%
%% Only for journal Encyclopedia
%\entrylink{The Link to this entry published on the encyclopedia platform.}

%%%%%%%%%%%%%%%%%%%%%%%%%%%%%%%%%%%%%%%%%%
%% Optional
%\abbreviations{The following abbreviations are used in this manuscript:\\

%\noindent 
%\begin{tabular}{@{}ll}
%MDPI & Multidisciplinary Digital Publishing Institute\\
%DOAJ & Directory of open access journals\\
%TLA & Three letter acronym\\
%LD & Linear dichroism
%\end{tabular}}
%
%%%%%%%%%%%%%%%%%%%%%%%%%%%%%%%%%%%%%%%%%%%
%%% Optional
%\appendixtitles{no} % Leave argument "no" if all appendix headings stay EMPTY (then no dot is printed after "Appendix A"). If the appendix sections contain a heading then change the argument to "yes".
%\appendixstart
%\appendix
%\section{}
%\subsection{}
%The appendix is an optional section that can contain details and data supplemental to the main text---for example, explanations of experimental details that would disrupt the flow of the main text but nonetheless remain crucial to understanding and reproducing the research shown; figures of replicates for experiments of which representative data are shown in the main text can be added here if brief, or as Supplementary Data. Mathematical proofs of results not central to the paper can be added as an appendix.
%
%\begin{specialtable}[H] 
%%\tablesize{\scriptsize}
%\caption{This is a table caption. Tables should be placed in the main text near to the first time they are~cited.\label{tab1}}
%%\tablesize{} % You can specify the fontsize here, e.g., \tablesize{\footnotesize}. If commented out \small will be used.
%\begin{tabular}{ccc}
%\toprule
%\textbf{Title 1}	& \textbf{Title 2}	& \textbf{Title 3}\\
%\midrule
%Entry 1		& Data			& Data\\
%Entry 2		& Data			& Data\\
%\bottomrule
%\end{tabular}
%\end{specialtable}
%
%\section{}
%All appendix sections must be cited in the main text. In the appendices, Figures, Tables, etc. should be labeled, starting with ``A''---e.g., Figure A1, Figure A2, etc. 

%%%%%%%%%%%%%%%%%%%%%%%%%%%%%%%%%%%%%%%%%%
\end{paracol}
\reftitle{References}

% Please provide either the correct journal abbreviation (e.g. according to the “List of Title Word Abbreviations” http://www.issn.org/services/online-services/access-to-the-ltwa/) or the full name of the journal.
% Citations and References in Supplementary files are permitted provided that they also appear in the reference list here. 

%=====================================
% References, variant A: external bibliography
%=====================================
%\externalbibliography{yes}
%\bibliography{your_external_BibTeX_file}

%=====================================
% References, variant B: internal bibliography
%=====================================
\begin{thebibliography}{999}
% Reference 1
\bibitem[Author1(year)]{ref-journal}
Author~1, T. The title of the cited article. {\em Journal Abbreviation} {\bf 2008}, {\em 10}, 142--149.
% Reference 2
\bibitem[Author2(year)]{ref-book1}
Author~2, L. The title of the cited contribution. In {\em The Book Title}; Editor1, F., Editor2, A., Eds.; Publishing House: City, Country, 2007; pp. 32--58.
% Reference 3
\bibitem[Author3(year)]{ref-book2}
Author 1, A.; Author 2, B. \textit{Book Title}, 3rd ed.; Publisher: Publisher Location, Country, 2008; pp. 154--196.
% Reference 4
\bibitem[Author4(year)]{ref-unpublish}
Author 1, A.B.; Author 2, C. Title of Unpublished Work. \textit{Abbreviated Journal Name} stage of publication (under review; accepted; in~press).
% Reference 5
\bibitem[Author5(year)]{ref-communication}
Author 1, A.B. (University, City, State, Country); Author 2, C. (Institute, City, State, Country). Personal communication, 2012.
% Reference 6
\bibitem[Author6(year)]{ref-proceeding}
Author 1, A.B.; Author 2, C.D.; Author 3, E.F. Title of Presentation. In Title of the Collected Work (if available), Proceedings of the Name of the Conference, Location of Conference, Country, Date of Conference; Editor 1, Editor 2, Eds. (if available); Publisher: City, Country, Year (if available); Abstract Number (optional), Pagination (optional).
% Reference 7
\bibitem[Author7(year)]{ref-thesis}
Author 1, A.B. Title of Thesis. Level of Thesis, Degree-Granting University, Location of University, Date of Completion.
% Reference 8
\bibitem[Author8(year)]{ref-url}
Title of Site. Available online: URL (accessed on Day Month Year).
\end{thebibliography}

% If authors have biography, please use the format below
%\section*{Short Biography of Authors}
%\bio
%{\raisebox{-0.35cm}{\includegraphics[width=3.5cm,height=5.3cm,clip,keepaspectratio]{Definitions/author1.pdf}}}
%{\textbf{Firstname Lastname} Biography of first author}
%
%\bio
%{\raisebox{-0.35cm}{\includegraphics[width=3.5cm,height=5.3cm,clip,keepaspectratio]{Definitions/author2.jpg}}}
%{\textbf{Firstname Lastname} Biography of second author}

% The following MDPI journals use author-date citation: Arts, Econometrics, Economies, Genealogy, Humanities, IJFS, JRFM, Laws, Religions, Risks, Social Sciences. For those journals, please follow the formatting guidelines on http://www.mdpi.com/authors/references
% To cite two works by the same author: \citeauthor{ref-journal-1a} (\citeyear{ref-journal-1a}, \citeyear{ref-journal-1b}). This produces: Whittaker (1967, 1975)
% To cite two works by the same author with specific pages: \citeauthor{ref-journal-3a} (\citeyear{ref-journal-3a}, p. 328; \citeyear{ref-journal-3b}, p.475). This produces: Wong (1999, p. 328; 2000, p. 475)

%%%%%%%%%%%%%%%%%%%%%%%%%%%%%%%%%%%%%%%%%%
%% for journal Sci
%\reviewreports{\\
%Reviewer 1 comments and authors’ response\\
%Reviewer 2 comments and authors’ response\\
%Reviewer 3 comments and authors’ response
%}
%%%%%%%%%%%%%%%%%%%%%%%%%%%%%%%%%%%%%%%%%%
\end{document}

